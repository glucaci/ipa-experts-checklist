\section{Bis zum Start der IPA}
Mit der Freigabe durch den Validierungsexperten wirst du gegenüber dem Kandidaten und dem Lehrbetrieb zum verantwortlichen Hauptexperten.\\Der Kandidat muss sich nun als dein Schützling auf dich verlassen können.\\Achte darauf, dass du für den Kandidaten erreichbar bist.

\begin{taskitemwithoutcomment}{Kommunikation}
  Nutze für die Kommunikation die History auf PKORG, damit die Nachvollziehbarkeit gewährleistet ist.\\Schreibe dort deine Nachrichten oder hinterlasse Notizen zu Telefongesprächen.\\Vermische keine Sachverhalte.\\Formuliere verständlich und anständig, sodass auch Dritte die Nachricht lesen dürfen (z. B. bei der Akteneinsicht und bei Rekursen).
\end{taskitemwithoutcomment}
\begin{taskitemwithoutcomment}{Begrüssungsnachricht auf PKOrg}
  Begrüsse den Kandidaten und seine verantwortliche Fachkraft mit einem Eintrag in der History.\\Teile den Beteiligten mit, wer du bist, wie du erreichbar bist.\\Und dass sie sich bei Fragen an dich wenden sollen.\\Informiere über deine nächsten Schritte.\\Und wünsche dem Kandidaten einen guten Start. 
\end{taskitemwithoutcomment}
\begin{taskitem}{Vereinbare die Besuchstermine}
  Vereinbare den 1. Besuchstermin mit der verantwortlichen Fachkraft und dem Kandidaten, am besten für den 2. oder 3. Tag der IPA.\\Vielleicht kannst du auch schon die weiteren Besuchstermine abmachen.\\Trage alle Besuchstermine in PKOrg ein.\\\textbf{Berücksichtige dabei auch die Verfügbarkeit vom NEX.}
\end{taskitem}
\begin{taskitemwithoutcomment}{IPA-Zeitplan}
  Fordere vom Kandidaten, dass er den IPA-Zeitplan vor dem ersten Besuch erstellt.\\Der Kandidat muss den IPA-Zeitplan als .pdf-Datei in die History auf PKOrg laden.\\Falls dies nicht möglich ist, soll er den IPA-Zeitplan per E-Mail an dich senden und du lädst ihn nach dem Besuchstermin auf PkOrg hoch.\\Der IPA-Zeitplan wird beim Besuchstermin angeschaut.
\end{taskitemwithoutcomment}

