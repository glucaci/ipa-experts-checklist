\section{Vorbereitungsphase}
Während der Vorbereitungsphase wird die Aufgabenbeschreibung vom Validierungsexperten betreut. Er ist in Kontakt mit dem Kandidaten, der verantwortlichen Fachkraft und dem Berufsbildner.
\begin{taskitemwithoutcomment}{Auswahl der IPA}
  \textbf{Wähle} die IPA sorgfältig \textbf{aus}.\\Es ist optimal und wird von der verantwortlichen Fachkraft geschätzt, wenn du als Experte einen Bezug zum Thema der Arbeit hast.\\Beachte zudem die \textbf{Ausstands-Regeln}.\\Prüfe, ob die Funktionen des Berufsbildners und der verantwortlichen Fachkraft richtig eingetragen sind.
\end{taskitemwithoutcomment}
\begin{taskitem}{Mithilfe bei der Validierung}
  Beteilige dich an der Validierung. Lese die Aufgabenstellung sorgfältig durch und überlege, ob du den Erfüllungsgrad und die Qualität der Facharbeit beurteilen kannst.\\Der Validierungsexperte ist dankbar für deine Hinweise im PKOrg-Validations-Dialog (nicht in der History).\\Beachte die Netiquette!
\end{taskitem}
\begin{taskitemwithoutcomment}{Deine Verfügbarkeit}
  Bist du während der Durchführung der IPA verfügbar?\\Überprüfe ob die in der Planung der IPA aufgeführten Termine für dich passen und du nicht zBsp. durch Ferien abwesend bist.\\Das Verschieben der IPA ist keine Option.
\end{taskitemwithoutcomment}
\begin{taskitemwithoutcomment}{Abgabe der IPA}
  Falls du die IPA wieder abgeben musst, dann sind wir froh, wenn du beim Abtausch mithilfst.\\Eventuell kannst du die IPA mit einem anderen Experten abtauschen.
\end{taskitemwithoutcomment}