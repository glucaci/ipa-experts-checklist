\section{Erster Besuchstermin durchführen}
Denke daran, dass du bei deinen Besuchen die kantonale Prüfungskommission vertrittst. Korrektes Auftreten, Höflichkeit, Geduld und Pünktlichkeit setzen wir für diese Arbeit voraus. Dein Amt erfordert aber auch Vertraulichkeit und Verschwiegenheit.\\Bereite Dich auf den Besuch vor. Mache während dem Besuch Notizen.

\begin{taskitem}{Vorstellungsrunde}
  Beginne das Gespräch mit einer Vorstellungsrunde. Gebe in erster Linie dem Kandidaten und der verantwortlichen Fachkraft Gelegenheit, sich vorzustellen. Beantworte aber auch Fragen zu deiner Person und zu deinem Arbeitgeber.
\end{taskitem}
\begin{taskitem}{Rollen}
  Stelle die Rollen der an der IPA beteiligten Personen (Kandidat, VF, HEX, NEX, CEX) vor. Erinnere die verantwortliche Fachkraft daran, dass du nun ihr Partner bei der Betreuung des Kandidaten bist.
\end{taskitem}
\begin{taskitemwithoutcomment}{Wichtige Dokumente}
  Haben die verantwortliche Fachkraft und der Kandidat die folgenden IPA-Dokumente gelesen?
  \begin{enumerate}
    \item Dokumente, welche beim Anmelden auf PKOrg bestätigt wurden
    \item QV-Leitfaden von \href{https://pk19.ch}{der Webseite der Prüfungskommission}
    \item QV-Termine von \href{https://pk19.ch}{der Webseite der Prüfungskommission}
  \end{enumerate}
  Diese Dokumente stellen im Kanton Zürich die aktuellen verbindlichen Vorgaben für den Inhalt und die Gestaltung des IPA-Berichts und können sich von anderen Kantonen unterscheiden.
\end{taskitemwithoutcomment}
\newpage
\begin{taskitem}{Bewertungskriterien}
  Bespreche die Bewertungskriterien, die Kriterien aus dem Kriterienkatalog und auch die 7 Kriterien aus der detaillierten Aufgabenstellung.\\Ein gemeinsames Verständnis ist Vorbedingung für eine reibungslose Bewertung.
  \begin{itemize}
    \item Standard Kriterienkatalog QV von \href{https://pk19.ch}{der Webseite der Prüfungskommission}
  \end{itemize}
  ~
\end{taskitem}
\begin{taskitemwithoutcomment}{Auftrag an die VF}
  Weise die verantwortliche Fachkraft daran hin, dass der Kandidat sorgfältig bei der Arbeit beobachtet werden muss. Und fordere die verantwortliche Fachkraft auf, ein Protokoll zu führen. Wichtig sind zum Beispiel Notizen zu Hilfestellungen der verantwortlichen Fachkraft während der IPA. Beobachtungen zum Vorgehen und dem Einsatz des Kandidaten sowie zu den Arbeitszeiten.
  \begin{itemize}
    \item Aufgaben-VF von \href{https://pk19.ch}{der Webseite der Prüfungskommission}
  \end{itemize}
\end{taskitemwithoutcomment}
\begin{taskitem}{Arbeitsplatz}
  Achte auf die Infrastruktur und das Umfeld.\\Arbeitet der Kandidat an seinem üblichen Arbeitsplatz? Ist ein ungestörtes Arbeiten möglich?\\Hat der Kandidat neben der IPA keine anderen Aufgaben zu erfüllen?
\end{taskitem}
\newpage
\begin{taskitemsmall}{Material und Vorarbeiten}
  Sind die Voraussetzungen für die Durchführung der IPA gemäss Aufgabenstellung erfüllt?\\Ist das Material, sind Installationen und Infrastruktur vorhanden und bereit?\\Sind die deklarierten Vorarbeiten erfolgreich abgeschlossen?
\end{taskitemsmall}
\begin{taskitemdoublelarge}{Gemeinsames Verständnis der Aufgabe}
  Hat der Kandidat die Aufgabe verstanden?\\Lasse dir die Aufgabenstellung in Form und Umfang nochmals vom Kandidaten bestätigen.\\Lese auch die Diskussion aus der Validierung nochmals durch. Hat es damals schon Fragen gegeben, die jetzt zu klären sind?
\end{taskitemdoublelarge}
\begin{taskitem}{Zeitplan besprechen}
  Kontrolliere, ob der Kandidaten den Zeitplan für die IPA erstellt und als .pdf-Datei auf PKOrg hochgeladen hat.\\Lasse Dir den Zeitplan durch den Kandidaten erklären.
\end{taskitem}
\newpage
\begin{taskitem}{Fragen und Probleme}
  Weise darauf hin, dass sich der Kandidat für alle Fragen und Eventualitäten (z.B. Probleme mit Hard- oder Software, aber auch bei Krankheit) an die VF und auch an dich wenden muss. Beachte dazu den Abschnitt weiter hinten in diesem Dokument: \enquote{Zwischenfälle: Pannen, Krankheit, Knapp, ...}).
\end{taskitem}
\begin{taskitem}{Nächste Schritte}
  Bespreche die weiteren Termine.\\Der zweite Besuchstermin sollte in der zweiten Hälfte der Facharbeit (Tag 7 oder 8) stattfinden.\\Der dritte Besuchstermin mit der Präsentation, Demonstration, Fachgespräch und Bewertung findet idealerweise etwa eine Woche bis 10 Tage nach Abgabe der IPA-Dokumentation statt.\\Nimm immer auf die Bedürfnisse des Betriebes und des Nebenexperten Rücksicht.\\Findet sich kein geeigneter gemeinsamer Termin mit dem NEX, soll dieser die Arbeit ab- und einem anderen Nebenexperten übergeben.\\Trage alle Termine immer auf PKOrg ein.\\Erkläre wiederum, wie man dich am besten erreichen kann.\\Weise darauf hin, dass die Kommunikation über PKOrg erfolgt.
\end{taskitem}