\section{Bis zur Präsentation}
Nehme dir Zeit, den IPA-Bericht gründlich durchzulesen. Denke daran, das ist ein Bericht einer ausgebildeten Fachkraft.\\Achte beim Lesen auf mögliche Themen für die Fachfragen.

\begin{taskitemwithoutcomment}{Bewertung Bereich A und B}
  Die Teile A und B werden anhand des abgegebenen Berichts, sowie den Eindrücken bei den Besuchsterminen bewertet und auf POrg eintragen.\\Begründe deine Bewertung nachvollziehbar, insbesondere, wenn Du nicht alle Punkte gibst. Es muss für Dritte klar erkennbar sein, wieso das Puntkemaximum nicht erreicht wurde.\\Lese die Bewertung der verantwortlichen Fachkraft und schaue, ob ihr derselben Meinung seid.
\end{taskitemwithoutcomment}
\begin{taskitemwithoutcomment}{Fachgespräch vorbereiten}
  Bereite acht bis zehn Gesprächsthemen für das Fachgespräch vor.\\Nutze das Formular auf PKOrg.\\Berücksichtige auch die Vorschläge der verantwortlichen Fachkraft. So kannst du ein ausgewogenes Fachgespräch vorbereiten.\\Notiere Beispiel-Antworten. Was erwartest du?\\Das Fachgespräch muss einen starken Bezug zur IPA haben. Es ist keine allgemeine Berufskundeprüfung. Zwar ist auch Wissen gefragt, aber immer im Zusammenhang mit der IPA. Das Fachgespräch soll aufzeigen, ob der Kandidat in seiner Fachrichtung und zu seiner IPA kompetent Auskunft geben kann.
\end{taskitemwithoutcomment}
\begin{taskitem}{Fachthemen aneignen}
  Informiere dich falls nötig zusätzlich zu den Themen der IPA. Besonders dann, wenn du mit der verantwortlichen Fachkraft oder dem Kandidaten an den Besuchsterminen zu wenig \enquote{fachsimpeln} konntest.
\end{taskitem}
\begin{taskitemwithoutcomment}{Rolle des Nebenexperten klären}
  Bei der Präsentation, Demonstration, dem Fachgespräch und der Bewertung ist üblicherweise ein zweiter Experte als Nebenexperte dabei. Er sorgt für vollständige Notizen und beteiligt sich am Fachgespräch und der Bewertung.\\Es gibt eine Checkliste für den Nebenexperten auf PKOrg.\\Falls kein Nebenexperte vorhanden ist, kannst du den dritten Besuch auch zu zweit mir der verantwortlichen Fachkraft durchführen.\\Erwartest Du Schwierigkeiten beim dritten Besuch, dann wende die an den Chefexperten, für die Suche nach einem Nebenexperten.
\end{taskitemwithoutcomment}
