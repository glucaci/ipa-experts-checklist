\section{Bewertung}
Die Bewertung muss unmittelbar im Anschluss an das Fachgespräch stattfinden.\\An der Sitzung nehmen nur noch die verantwortliche Fachkraft und die Experten teil.\\In der Regel wird die Bewertung eine gute Stunde dauern.\\Der Hauptexperte leitet das Gespräch.\\Das Ziel ist es, eine Note zu finden, mit welcher die Experten und die verantwortliche Fachkraft einverstanden sind.

\begin{taskitem}{Gesamteindruck}
  Frage die verantwortliche Fachkraft zuerst nach dem Gesamteindruck.
\end{taskitem}
\begin{taskitem}{Teil C: Präsentation, Demonstration und Fachgespräch auswerten}
  Es macht Sinn, mit der Bewertung der Präsentation, Demonstration und Fachgesprächs zu beginnen.\\Die effektiv geführten Dialoge des Fachgesprächs sind zu gruppieren und zu den vorgesehenen Fachgesprächskriterien zuzuordnen.\\Spontane Fragen bedingen unter Umständen eine Anpassung der vorbereiteten Kriterien. Ergänze die Notizen direkt im Formular zu den Fachfragen.\\Es ist wichtig, dass Dritte die Bewertung der Fachfragen nachvollziehen können.
\end{taskitem}
\newpage
\begin{taskitem}{Bewertung Teil A und B vergleichen und konsolidieren}
  Anschliessend besprecht ihr die Bewertung von Teil A und B.\\Fokussiere dich auf unterschiedliche Bewertungen zwischen verantwortlicher Fachkraft und Hauptexperte.\\Vermeide Grundsatzdiskussionen und ausschweifendes Fachsimpeln.\\Frage den Nebenexperten nach seiner Meinung.
\end{taskitem}
\begin{taskitemwithoutcomment}{Bewertungsraster auf PKOrg ergänzen}
  Fülle das Bewertungsraster in PKOrg aus. Beachte, dass für jedes Kriterium eine nachvollziehbare Begründung eingetragen werden muss. (Bsp. \enquote{unvollständig}: was fehlt oder ist nicht korrekt?)\\Denke daran: die Punkte repräsentieren keine Note, sondern eine Gütestufe des Kriteriums. Erst die Summe aller Punkte ergibt einen Notenvorschlag.
\end{taskitemwithoutcomment}
\begin{taskitemwithoutcomment}{Einigkeit}
  Der Notenvorschlag erscheint im PKOrg so bald alle Gütestufen und Bemerkungen eingegeben sind.\\Schaut kurz zurück: entspricht das Resultat den Erwartungen?\\Bei Uneinigkeit kann dies in PKOrg mit einem Häkchen vermerkt und der Entscheid der Notenkonferenz überlassen werden.\\Bitte diese Möglichkeit nur benutzen, wenn es wirklich nicht anders geht.
\end{taskitemwithoutcomment}
\begin{taskitem}{Rechtzeitige Abgabe des Berichts}
  Stelle fest, ob der Bericht rechtzeitig hochgeladen wurde. Ansonsten gibt es einen Notenabzug nach Vorgaben der Prüfungsleitung.\\Du musst in diesem Fall zwingend den CEX informieren!
\end{taskitem}
\begin{taskitemwithoutcomment}{Signatur der Bewertung}
  Die Bewertung muss von der verantwortlichen Fachkraft und vom Nebenexperten signiert werden.
\end{taskitemwithoutcomment}
\newpage
\begin{taskitemwithoutcomment}{Unterlagen und Notizen auf PKOrg hochladen}
  Lade alle deine Notizen in den Dokumentenpool des Kandidaten auf PKOrg hoch. Achte darauf, dass die verantwortliche Fachkraft und der Nebenexperte dies auch gleich erledigen.
\end{taskitemwithoutcomment}
\begin{taskitemwithoutcomment}{Notenvorschlag vertraulich behandeln}
  Mache die verantwortliche Fachkraft darauf aufmerksam, dass er den Notenvorschlag dem Kandidaten nicht mitteilen darf, weil dieser auch im Nachhinein (beim Quervergleich) durch die Prüfungsleitung verändert werden kann.
\end{taskitemwithoutcomment}
\begin{taskitemwithoutcomment}{Vertraulichkeit}
  Mache die verantwortliche Fachkraft darauf aufmerksam, dass alle Unterlagen vertraulich behandelt werden und im Kreis der Experten der Prüfungskommission verbleiben.
\end{taskitemwithoutcomment}
\begin{taskitemwithoutcomment}{Danksagung}
  Bedanke dich bei der verantwortlichen Fachkraft für ihren Einsatz und ermuntere sie, ebenfalls als Experte mitzuwirken. Weitere Informationen dazu können auf \href{https://pk19.ch}{der Webseite der Prüfungsorganisation} gefunden werden.
\end{taskitemwithoutcomment}
\begin{taskitem}{Austausch mit Nebenexperte}
  Nach Verabschiedung der verantwortlichen Fachkraft nutze die Gelegenheit und besprechen den Ablauf des Prüfungstages kritisch. Beispielfragen: \enquote{Wie war das Verhältnis zum Kandidaten? Wie war der Führungsstil? Gab es Situationen, in der der Kandidat sich unwohl fühlte? War das Fachgespräch flüssig und angemessen? Wo hätte man mehr Aufmerksamkeit schenken können? Strenger/weniger streng reagieren?}
\end{taskitem}
